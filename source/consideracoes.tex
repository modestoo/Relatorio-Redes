\chapter[Considerações Finais]{Considerações Finais}
\label{chap:consideracoes}

	Com base nos resultados obtidos, é possível inferir a importância da utilização de ferramentas para monitoamento de redes. Pois, a utilização de software \emph{firewall} não é suficiente para prevenir-se de todos os ataques. Além disso, sistemas de \emph{firewall} mal configurados podem expor toda a rede a ataques até simples, como o \emph{brute force} realizado nos experimentos deste projeto.
 
	Ressalta-se também a importância de ter uma boa equipe para administração da rede, dos sistemas e das políticas de segurança. Essa caracteristica pode reduzir consideravelmente os ataques típicos à rede, uma vez que essa equipe configure corretamente as ferramentas de segurança de redes.

	Ainda conclui-se que as ferramentas IDS/IPS são extremamente essenciais à uma rede, principalmente quando são envolvidos dados confidênciais ao qual requerem segurança extra para não serem violados. E, mesmo com pontos negativos apresentados por estas ferramentas, tais como detecção de tráfego de falsos positivos e falsos negativos, vale a pena sua utilização. Pois, muito melhor uma ferramenta cuja possíveis ataques à rede são detectados e inibidos, ao passo de uma rede que não previne nada.
	
	Para trabalhos futuros, pode-se adicionar novas instâncias de ataques, principalmente os mais complexos como um DDoS, a fim de averiguar a eficiência das ferramentas utilizadas e monitorar o comportamento das regras estabelecidas.