\chapter[Introdução]{Introdução}
\label{chap:introducao}
	
		Ferramentas que são capazes de detectar ataques à rede ou a um \textit{host} são definidas como IDS ou Sistema de Detecção de Intrusos. Tal mecanismo de segurança pode trabalhar de modo colaborativo com ferramentas que possuem a habilidade de filtrar tráfegos suspeitos, conhecidos como IPS  ou Sistema de Prevenção de Intrusos \cite{Kurose}.
		
		O presente trabalho tem como objetivo simular a implantação de tais mecanismos de segurança (IDS/IPS) para o monitoramento de redes, demonstrando como tais tipos de ferramentas podem ser implantadas  e configuradas em uma rede afim de detectar ataques e reagir aos mesmos.
		
		Para esclarecimento teórico sobre o assunto e descrição das simulações realizadas, foram destinados cinco capítulos do presente documento. O primeiro capítulo trata de definições sobre ferramentas IDS/IPS, bem como suas características, classificações e descrição das ferramentas utilizadas para a realização deste projeto de disciplina. O segundo capítulo trata da arquitetura de rede estabelecida para a simulação da implantação de tais ferramentas, exibindo seus componentes, esquema de infraestrutura, bem como limitações encontradas. O terceiro capítulo trata sobre os principais tipos de ataques que podem ser praticados contra uma rede e descreve quais destes foram planejados para a simulação. A simulação de utilização de regras no sistema IDS/IPS utilizado para o desenvolvimento do presente trabalho, bem como dos ataques planejados são descritos nos dois capítulos subsequentes.
		
		A metodologia utilizada consistiu em pesquisa bibligráfica sobre o assunto e revisão sistemática simples sobre trabalhos práticos relacionados, afim de traçar estratégias viáveis para a execução da simulação prática realizada. Foram incluídos livros, artigos, dissertações e manuais e guias disponibilizados na Internet e em meio impresso.