\chapter{Sistema de Detecção de Intrusões}
\label{chap:IDS}
	
	Nesta seção serão abordados tópicos ....

	Com o avanço dos computadores e tecnologias que o perpetuam, uma gama de benefícios para os usuários destas tecnologias foi evidenciadas. No entanto, paralelo a isso, o mundo cibernético torna-se cada vez mais perigoso, uma vez que este avanço proporciona também o aumento de ataques a redes de computadores. No entanto, previnir-se ainda não é sucifiente, pois o número de ataques e sua complexidade tendem a crescer além das ferramentas utilizadas para tais ataques ficaram cada vez mais automatizadas.

	Então, uma grande necessidade surge: como garantir um comportamento livres de vulnerabilidade, falhas e ataques?

	Neste contexto, surge no início dos anos 80 os primeiros conceitos de métodos de detecção de instruções no qual ganha espaço ao longo dos anos seguintes.

	Atualmente o IDS é uma ferramenta de (..........)


	\section{Tipos de IDS}
	\label{sec:IDS_Tipos}

		TO DO.

	\section{Relação IDS e Firewall}
	\label{sec:IDS_Firewall}

		TO DO.

	\section{Vantagens e Limitações do IDS}
	\label{sec:IDS_VeL}

		TO DO.

	\section{Levantamento das Ferramentas IDS}
	\label{sec:IDS_Ferramentas}

		TO DO.

